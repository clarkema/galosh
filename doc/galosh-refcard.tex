\documentclass[10pt,landscape,a4paper]{article}
\usepackage{pslatex}
\usepackage{multicol}

% Turn off header and footer
\pagestyle{empty}

%\setlength{\leftmargin}{0.75in}
\setlength{\oddsidemargin}{-0.75in}
\setlength{\evensidemargin}{-0.75in}
\setlength{\textwidth}{10.5in}

\setlength{\topmargin}{-0.2in}
\setlength{\textheight}{7.4in}
\setlength{\headheight}{0in}
\setlength{\headsep}{0in}

\pdfpageheight\paperheight
\pdfpagewidth\paperwidth

% Redefine section commands to use less space
\makeatletter
\renewcommand\section{\@startsection{section}{1}{0mm}%
                                     {-24pt}% \@plus -12pt \@minus -6pt}%
                                     {0.5ex}%
                                {\large\bfseries}}
\makeatother

% Don't print section numbers
\setcounter{secnumdepth}{0}

\setlength{\parindent}{0pt}
\setlength{\parskip}{0pt}

\newcommand{\code}{\texttt}
\newcommand{\bcode}[1]{\texttt{#1}}
\newcommand\F{\code{FALSE}}
\newcommand\T{\code{TRUE}}

\newcommand{\describe}[1]{\begin{description}{#1}\end{description}}

\begin{document}

\footnotesize
\begin{multicols*}{3}

% multicol parameters
% These lengths are set only within the two main columns
%\setlength{\columnseprule}{0.25pt}
\setlength{\premulticols}{1pt}
\setlength{\postmulticols}{1pt}
\setlength{\multicolsep}{1pt}
\setlength{\columnsep}{2pt}

\begin{center}
     \Large{\textbf{Galosh Quick Reference}} \\
\end{center}

\section{galosh log}
\everypar={\hangindent=9mm}

New log entry: \code{<CALL> [<RX-REPORT>] [<TX-REPORT>]} \\
\code{RX-REPORT} and \code{TX-REPORT} can both be skipped, in which case
a sensible default is chosen depending on your mode.  For phone modes, a
single digit $x$ is taken to mean $5x$.  For tone modes, a single digit is
taken to mean $5x9$.

\code{:set qrg 14260000} sets a new frequency, in Hertz.

\code{:set mode SSB} sets a new operating mode.

\code{:set iota AN-001} updates your IOTA reference.

\code{:d <ID>} deletes a QSO (backup stored in .galosh/log.attic)

\code{:q} quits.

\subsection{QSO Editing Mode}

\code{f} to mark the QSO for followup.

\code{C} to edit the other station's call.

\code{n} to edit the other operator's name.

\code{g} to edit the other station's gridsquare.

\code{i} to edit the other station's IOTA reference.

\code{c} to add a comment.

\code{ENTER} to log the QSO.

\code{ESC} to cancel.

\section{galosh qsl}

\code{i}: edit ITU zone.

\code{I}: edit IOTA reference.

\code{D}: edit DXCC.

\code{G}: edit grid reference.

\code{o}: edit country.

\code{r}: mark QSL received.

\code{t}: mark QSL queued for sending.

\code{v}: merge remote station details from qrz.com.

\code{a}: combined effects of \code{r}, \code{t} and \code{v}.

\code{A}: edit call.

\code{m}: mark QSO.

\code{p}: print QSL for QSO.

\code{P}: preview QSL for QSO.

\end{multicols*}
\end{document}
